
\documentclass[a4paper, 10pt]{article}
\usepackage[spanish]{babel}
\usepackage[utf8]{inputenc}
\usepackage[top=1 in, left=1 in, right=1 in, bot=1 in]{geometry}
\usepackage{amsmath, amsthm, amsfonts}
\usepackage{graphics}
\usepackage{float}
\usepackage{epsfig}
\usepackage{amssymb}
\usepackage{latexsym}
\usepackage{newlfont}
\usepackage{epstopdf}
\usepackage{amsthm}
\usepackage{epsfig}
\usepackage{caption}
\usepackage{multirow}
\usepackage[colorlinks]{hyperref}
\usepackage[x11names,table]{xcolor}
\usepackage{graphics}
\usepackage{wrapfig}
\usepackage[rflt]{floatflt}
\usepackage{multicol}
\usepackage{longtable,multirow,booktabs}
\usepackage{listings} \lstset {language = Python, basicstyle=\bfseries\ttfamily, keywordstyle = \color{blue}, commentstyle = \bf\color{green}, backgroundcolor = \color{gray!5}, stringstyle = \color{yellow!5}}
\usepackage{titlesec}

% \titleformat{\section}{\filcenter}{}{8pt}{}

\setcounter{secnumdepth}{0}

\title{Sudoku Nonominó en Haskell}
\author{Alexander A. González Fertel C-412\\
		\href{mailto:a.fertel@estudiantes.matcom.uh.cu}{a.fertel@estudiantes.matcom.uh.cu}}
\date{}

\begin{document}
	\maketitle

	\section{Nonominó}
	Los nonominos se representan como un tipo {\itshape Record} con dos propiedades: {\bfseries nid},
	que se utiliza a la hora de discernir un nonominó de otro y {\bfseries points}, que es una
	lista de tuplas donde el primer elemento es un entero representando qué dígito hay en la casilla
	representada por el segundo elemento que es una tupla $x, y$. La idea seguida
	es que dicha lista de 'puntos' representa desplazamientos con respecto a una casilla del tablero,
	dichos desplazamientos comienzan por la tupla 'raíz', es decir, el inicio de la lista de puntos
	debe ser el punto con desplazamiento $(0, 0)$, que además debe corresponder a la casilla más arriba
	a la izquierda,
	se implementó una función que dado un nonominó y un punto $x, y$, devuelve un nuevo nonominó
	con el desplazamiento dado a partir de este punto. 

	\section{Sudoku}
	El sudoku esta representado por un tipo {\itshape Board}, que es simplemente una lista de
	{\itshape Square}. El tipo {\itshape Square} está representado por una fila, una columna,
	un identificador de nonomino ({\bfseries nid}) y un {\itshape Either [Int] Int}, respectivamente:
	
	\begin{itemize}
		\item $row :: Int$	
		\item $col :: Int$	
		\item $nono :: Maybe Int$	
		\item $status :: Either [Int] Int$	
	\end{itemize}

	La propiedad {\bfseries nono} es de tipo {\itshape Maybe Int} ya que inicialmente, las
	casillas del tablero no tienen asociadas un nonominó; el entero es el identificador del nonominó.
	La propiedad {\bfseries status} define en que 'estado' está dicha casilla: si la casilla
	tiene un valor definido, entonces es {\itshape Right Int}, si no, entonces es una lista de
	posibles dígitos {\bfseries Left [Int]}.

	\section{Principales Ideas}
	\subsection{Ensamblado}
	Para ensamblar los nonominós se toman todas las permutaciones de la lista que contiene
	los naturales del uno al ocho y se intentan colocar los nonominós, que tienen un orden
	predefinido, como dicta dicha permutación. Para colocar el nonominó correspondiente,
	se busca la casilla más arriba a la izquierda que no está ocupada todavía por un nonominó y
	se intenta colocar dicho nonominó ahí.

	\subsection{Resolver}
	Luego de ensamblados los nonominós, queda como resultado un tablero de sudoku donde a diferencia
	del sudoku clásico, en ved de tener cuadraditos de $3 x 3$, contamos con nonominós, por tanto,
	se pudiera proceder de manera análoga a como se procedería en el clásico, solo que ahora, no pueden
	ocurrir dos dígitos dentro de un mismo nonominó. La solución es ordenar de forma no decreciente 
	por longitud de las posibilidades
	a las casillas del tablero, de forma tal que siempre se traten de escribir primero los dígitos
	de las casillas en las que solo hay una posibilidad, reduciendo notablemente la complejidad
	temporal del algoritmo. Por lo demás, es un backtrack bastante simple.

	\section{Función de inicio}
	La función de inicio es $main :: IO ()$, que se encuentra en el archivo $Main.hs$
	y se puede encontrar implementada con 1 ejemplo de uso, para la prueba de diferentes
	nonominós, simplemente se cambia la lista en el archivo $Test.hs$ y se vuelve a ejecutar la función.
	Esta llama a la función que ensambla la lista de nonominós y luego llama a la función que
	resuelve el sudoku resultante. En la salida se muestran los nonominós ensamblados, el tablero
	sin resolver y el tablero resuelto.

	\section{Notas}
	Se hace uso de las mónadas {\bfseries []} (lista) y {\bfseries IO}.
	Se provee de comentarios en todas las partes importantes del código.
	Se intenta mostrar en consola con colores, pero por alguna extraña razón muestra caracteres
	desconocidos.

\end{document}
